% Options for packages loaded elsewhere
\PassOptionsToPackage{unicode}{hyperref}
\PassOptionsToPackage{hyphens}{url}
%
\documentclass[
]{article}
\usepackage{amsmath,amssymb}
\usepackage{lmodern}
\usepackage{iftex}
\ifPDFTeX
  \usepackage[T1]{fontenc}
  \usepackage[utf8]{inputenc}
  \usepackage{textcomp} % provide euro and other symbols
\else % if luatex or xetex
  \usepackage{unicode-math}
  \defaultfontfeatures{Scale=MatchLowercase}
  \defaultfontfeatures[\rmfamily]{Ligatures=TeX,Scale=1}
\fi
% Use upquote if available, for straight quotes in verbatim environments
\IfFileExists{upquote.sty}{\usepackage{upquote}}{}
\IfFileExists{microtype.sty}{% use microtype if available
  \usepackage[]{microtype}
  \UseMicrotypeSet[protrusion]{basicmath} % disable protrusion for tt fonts
}{}
\makeatletter
\@ifundefined{KOMAClassName}{% if non-KOMA class
  \IfFileExists{parskip.sty}{%
    \usepackage{parskip}
  }{% else
    \setlength{\parindent}{0pt}
    \setlength{\parskip}{6pt plus 2pt minus 1pt}}
}{% if KOMA class
  \KOMAoptions{parskip=half}}
\makeatother
\usepackage{xcolor}
\usepackage[margin=1in]{geometry}
\usepackage{color}
\usepackage{fancyvrb}
\newcommand{\VerbBar}{|}
\newcommand{\VERB}{\Verb[commandchars=\\\{\}]}
\DefineVerbatimEnvironment{Highlighting}{Verbatim}{commandchars=\\\{\}}
% Add ',fontsize=\small' for more characters per line
\usepackage{framed}
\definecolor{shadecolor}{RGB}{248,248,248}
\newenvironment{Shaded}{\begin{snugshade}}{\end{snugshade}}
\newcommand{\AlertTok}[1]{\textcolor[rgb]{0.94,0.16,0.16}{#1}}
\newcommand{\AnnotationTok}[1]{\textcolor[rgb]{0.56,0.35,0.01}{\textbf{\textit{#1}}}}
\newcommand{\AttributeTok}[1]{\textcolor[rgb]{0.77,0.63,0.00}{#1}}
\newcommand{\BaseNTok}[1]{\textcolor[rgb]{0.00,0.00,0.81}{#1}}
\newcommand{\BuiltInTok}[1]{#1}
\newcommand{\CharTok}[1]{\textcolor[rgb]{0.31,0.60,0.02}{#1}}
\newcommand{\CommentTok}[1]{\textcolor[rgb]{0.56,0.35,0.01}{\textit{#1}}}
\newcommand{\CommentVarTok}[1]{\textcolor[rgb]{0.56,0.35,0.01}{\textbf{\textit{#1}}}}
\newcommand{\ConstantTok}[1]{\textcolor[rgb]{0.00,0.00,0.00}{#1}}
\newcommand{\ControlFlowTok}[1]{\textcolor[rgb]{0.13,0.29,0.53}{\textbf{#1}}}
\newcommand{\DataTypeTok}[1]{\textcolor[rgb]{0.13,0.29,0.53}{#1}}
\newcommand{\DecValTok}[1]{\textcolor[rgb]{0.00,0.00,0.81}{#1}}
\newcommand{\DocumentationTok}[1]{\textcolor[rgb]{0.56,0.35,0.01}{\textbf{\textit{#1}}}}
\newcommand{\ErrorTok}[1]{\textcolor[rgb]{0.64,0.00,0.00}{\textbf{#1}}}
\newcommand{\ExtensionTok}[1]{#1}
\newcommand{\FloatTok}[1]{\textcolor[rgb]{0.00,0.00,0.81}{#1}}
\newcommand{\FunctionTok}[1]{\textcolor[rgb]{0.00,0.00,0.00}{#1}}
\newcommand{\ImportTok}[1]{#1}
\newcommand{\InformationTok}[1]{\textcolor[rgb]{0.56,0.35,0.01}{\textbf{\textit{#1}}}}
\newcommand{\KeywordTok}[1]{\textcolor[rgb]{0.13,0.29,0.53}{\textbf{#1}}}
\newcommand{\NormalTok}[1]{#1}
\newcommand{\OperatorTok}[1]{\textcolor[rgb]{0.81,0.36,0.00}{\textbf{#1}}}
\newcommand{\OtherTok}[1]{\textcolor[rgb]{0.56,0.35,0.01}{#1}}
\newcommand{\PreprocessorTok}[1]{\textcolor[rgb]{0.56,0.35,0.01}{\textit{#1}}}
\newcommand{\RegionMarkerTok}[1]{#1}
\newcommand{\SpecialCharTok}[1]{\textcolor[rgb]{0.00,0.00,0.00}{#1}}
\newcommand{\SpecialStringTok}[1]{\textcolor[rgb]{0.31,0.60,0.02}{#1}}
\newcommand{\StringTok}[1]{\textcolor[rgb]{0.31,0.60,0.02}{#1}}
\newcommand{\VariableTok}[1]{\textcolor[rgb]{0.00,0.00,0.00}{#1}}
\newcommand{\VerbatimStringTok}[1]{\textcolor[rgb]{0.31,0.60,0.02}{#1}}
\newcommand{\WarningTok}[1]{\textcolor[rgb]{0.56,0.35,0.01}{\textbf{\textit{#1}}}}
\usepackage{graphicx}
\makeatletter
\def\maxwidth{\ifdim\Gin@nat@width>\linewidth\linewidth\else\Gin@nat@width\fi}
\def\maxheight{\ifdim\Gin@nat@height>\textheight\textheight\else\Gin@nat@height\fi}
\makeatother
% Scale images if necessary, so that they will not overflow the page
% margins by default, and it is still possible to overwrite the defaults
% using explicit options in \includegraphics[width, height, ...]{}
\setkeys{Gin}{width=\maxwidth,height=\maxheight,keepaspectratio}
% Set default figure placement to htbp
\makeatletter
\def\fps@figure{htbp}
\makeatother
\setlength{\emergencystretch}{3em} % prevent overfull lines
\providecommand{\tightlist}{%
  \setlength{\itemsep}{0pt}\setlength{\parskip}{0pt}}
\setcounter{secnumdepth}{-\maxdimen} % remove section numbering
\ifLuaTeX
  \usepackage{selnolig}  % disable illegal ligatures
\fi
\IfFileExists{bookmark.sty}{\usepackage{bookmark}}{\usepackage{hyperref}}
\IfFileExists{xurl.sty}{\usepackage{xurl}}{} % add URL line breaks if available
\urlstyle{same} % disable monospaced font for URLs
\hypersetup{
  pdftitle={R Notebook},
  hidelinks,
  pdfcreator={LaTeX via pandoc}}

\title{R Notebook}
\author{}
\date{\vspace{-2.5em}}

\begin{document}
\maketitle

\hypertarget{chapter-2}{%
\section{Chapter 2}\label{chapter-2}}

\hypertarget{load-the-data}{%
\subsection{Load the data}\label{load-the-data}}

\begin{Shaded}
\begin{Highlighting}[]
\NormalTok{wine\_data }\OtherTok{=} \FunctionTok{read.csv}\NormalTok{(}\StringTok{"winequality{-}all.csv"}\NormalTok{, }\AttributeTok{header =}\NormalTok{ T, }\AttributeTok{na.strings =} \StringTok{"?"}\NormalTok{, }\AttributeTok{stringsAsFactors =}\NormalTok{ T)}
\NormalTok{wine\_data}\SpecialCharTok{$}\NormalTok{color }\OtherTok{\textless{}{-}} \FunctionTok{factor}\NormalTok{(wine\_data}\SpecialCharTok{$}\NormalTok{color, }\AttributeTok{levels=}\FunctionTok{c}\NormalTok{(}\DecValTok{1}\NormalTok{, }\DecValTok{0}\NormalTok{), }\AttributeTok{labels=}\FunctionTok{c}\NormalTok{(}\StringTok{\textquotesingle{}R\textquotesingle{}}\NormalTok{, }\StringTok{\textquotesingle{}W\textquotesingle{}}\NormalTok{))}

\NormalTok{drop }\OtherTok{\textless{}{-}} \FunctionTok{c}\NormalTok{(}\StringTok{"color"}\NormalTok{)}
\NormalTok{wine\_data }\OtherTok{=}\NormalTok{ wine\_data[,}\SpecialCharTok{!}\NormalTok{(}\FunctionTok{names}\NormalTok{(wine\_data) }\SpecialCharTok{\%in\%}\NormalTok{ drop)]}
\end{Highlighting}
\end{Shaded}

\hypertarget{capping}{%
\subsection{Capping}\label{capping}}

\begin{Shaded}
\begin{Highlighting}[]
\FunctionTok{library}\NormalTok{(scales)}
\end{Highlighting}
\end{Shaded}

\begin{verbatim}
## Warning: package 'scales' was built under R version 4.2.2
\end{verbatim}

\begin{Shaded}
\begin{Highlighting}[]
\NormalTok{wine\_data}\SpecialCharTok{$}\NormalTok{fixed.acidity }\OtherTok{\textless{}{-}} \FunctionTok{squish}\NormalTok{(wine\_data}\SpecialCharTok{$}\NormalTok{fixed.acidity, }\FunctionTok{quantile}\NormalTok{(wine\_data}\SpecialCharTok{$}\NormalTok{fixed\_acidity, }\FunctionTok{c}\NormalTok{(.}\DecValTok{05}\NormalTok{, .}\DecValTok{95}\NormalTok{)))}

\NormalTok{wd\_capped }\OtherTok{\textless{}{-}}\NormalTok{ wine\_data}

\NormalTok{varlist }\OtherTok{\textless{}{-}} \FunctionTok{names}\NormalTok{(wine\_data)}
\CommentTok{\# varlist \textless{}{-} varlist[{-}length(varlist)]}

\ControlFlowTok{for}\NormalTok{ (i }\ControlFlowTok{in}\NormalTok{ varlist) \{}
\NormalTok{    var }\OtherTok{\textless{}{-}} \FunctionTok{eval}\NormalTok{(}\FunctionTok{parse}\NormalTok{(}\AttributeTok{text =} \FunctionTok{paste0}\NormalTok{(}\StringTok{"wine\_data$"}\NormalTok{, i)))}
\NormalTok{    var }\OtherTok{\textless{}{-}} \FunctionTok{squish}\NormalTok{(var, }\FunctionTok{quantile}\NormalTok{(var, }\FunctionTok{c}\NormalTok{(.}\DecValTok{05}\NormalTok{, .}\DecValTok{95}\NormalTok{)))}
\NormalTok{\}}


\NormalTok{wd\_capped}\SpecialCharTok{$}\NormalTok{fixed.acidity }\OtherTok{\textless{}{-}} \FunctionTok{squish}\NormalTok{(wd\_capped}\SpecialCharTok{$}\NormalTok{fixed.acidity, }\FunctionTok{quantile}\NormalTok{(wd\_capped}\SpecialCharTok{$}\NormalTok{fixed\_acidity, }\FunctionTok{c}\NormalTok{(.}\DecValTok{05}\NormalTok{, .}\DecValTok{95}\NormalTok{)))}
\NormalTok{wd\_capped}\SpecialCharTok{$}\NormalTok{volatile.acidity }\OtherTok{\textless{}{-}} \FunctionTok{squish}\NormalTok{(wd\_capped}\SpecialCharTok{$}\NormalTok{volatile.acidity, }\FunctionTok{quantile}\NormalTok{(wd\_capped}\SpecialCharTok{$}\NormalTok{volatile.acidity, }\FunctionTok{c}\NormalTok{(.}\DecValTok{05}\NormalTok{, .}\DecValTok{95}\NormalTok{)))    }
\NormalTok{wd\_capped}\SpecialCharTok{$}\NormalTok{citric.acid}\OtherTok{\textless{}{-}} \FunctionTok{squish}\NormalTok{(wd\_capped}\SpecialCharTok{$}\NormalTok{citric.acid, }\FunctionTok{quantile}\NormalTok{(wd\_capped}\SpecialCharTok{$}\NormalTok{citric.acid, }\FunctionTok{c}\NormalTok{(.}\DecValTok{05}\NormalTok{, .}\DecValTok{95}\NormalTok{)))          }
\NormalTok{wd\_capped}\SpecialCharTok{$}\NormalTok{residual.sugar }\OtherTok{\textless{}{-}} \FunctionTok{squish}\NormalTok{(wd\_capped}\SpecialCharTok{$}\NormalTok{residual.sugar, }\FunctionTok{quantile}\NormalTok{(wd\_capped}\SpecialCharTok{$}\NormalTok{residual.sugar, }\FunctionTok{c}\NormalTok{(.}\DecValTok{05}\NormalTok{, .}\DecValTok{95}\NormalTok{)))       }
\NormalTok{wd\_capped}\SpecialCharTok{$}\NormalTok{chlorides }\OtherTok{\textless{}{-}} \FunctionTok{squish}\NormalTok{(wd\_capped}\SpecialCharTok{$}\NormalTok{chlorides, }\FunctionTok{quantile}\NormalTok{(wd\_capped}\SpecialCharTok{$}\NormalTok{chlorides, }\FunctionTok{c}\NormalTok{(.}\DecValTok{05}\NormalTok{, .}\DecValTok{95}\NormalTok{)))            }
\NormalTok{wd\_capped}\SpecialCharTok{$}\NormalTok{free.sulfur.dioxide }\OtherTok{\textless{}{-}} \FunctionTok{squish}\NormalTok{(wd\_capped}\SpecialCharTok{$}\NormalTok{free.sulfur.dioxide, }\FunctionTok{quantile}\NormalTok{(wd\_capped}\SpecialCharTok{$}\NormalTok{free.sulfur.dioxide, }\FunctionTok{c}\NormalTok{(.}\DecValTok{05}\NormalTok{, .}\DecValTok{95}\NormalTok{)))  }
\NormalTok{wd\_capped}\SpecialCharTok{$}\NormalTok{total.sulfur.dioxide }\OtherTok{\textless{}{-}} \FunctionTok{squish}\NormalTok{(wd\_capped}\SpecialCharTok{$}\NormalTok{total.sulfur.dioxide, }\FunctionTok{quantile}\NormalTok{(wd\_capped}\SpecialCharTok{$}\NormalTok{total.sulfur.dioxide, }\FunctionTok{c}\NormalTok{(.}\DecValTok{05}\NormalTok{, .}\DecValTok{95}\NormalTok{))) }
\NormalTok{wd\_capped}\SpecialCharTok{$}\NormalTok{density }\OtherTok{\textless{}{-}} \FunctionTok{squish}\NormalTok{(wd\_capped}\SpecialCharTok{$}\NormalTok{density, }\FunctionTok{quantile}\NormalTok{(wd\_capped}\SpecialCharTok{$}\NormalTok{density, }\FunctionTok{c}\NormalTok{(.}\DecValTok{05}\NormalTok{, .}\DecValTok{95}\NormalTok{)))}
\NormalTok{wd\_capped}\SpecialCharTok{$}\NormalTok{pH }\OtherTok{\textless{}{-}} \FunctionTok{squish}\NormalTok{(wd\_capped}\SpecialCharTok{$}\NormalTok{pH, }\FunctionTok{quantile}\NormalTok{(wd\_capped}\SpecialCharTok{$}\NormalTok{pH, }\FunctionTok{c}\NormalTok{(.}\DecValTok{05}\NormalTok{, .}\DecValTok{95}\NormalTok{)))                  }
\NormalTok{wd\_capped}\SpecialCharTok{$}\NormalTok{sulphates }\OtherTok{\textless{}{-}} \FunctionTok{squish}\NormalTok{(wd\_capped}\SpecialCharTok{$}\NormalTok{sulphates, }\FunctionTok{quantile}\NormalTok{(wd\_capped}\SpecialCharTok{$}\NormalTok{sulphates, }\FunctionTok{c}\NormalTok{(.}\DecValTok{05}\NormalTok{, .}\DecValTok{95}\NormalTok{)))           }
\NormalTok{wd\_capped}\SpecialCharTok{$}\NormalTok{alcohol }\OtherTok{\textless{}{-}} \FunctionTok{squish}\NormalTok{(wd\_capped}\SpecialCharTok{$}\NormalTok{alcohol, }\FunctionTok{quantile}\NormalTok{(wd\_capped}\SpecialCharTok{$}\NormalTok{alcohol, }\FunctionTok{c}\NormalTok{(.}\DecValTok{05}\NormalTok{, .}\DecValTok{95}\NormalTok{)))   }
\NormalTok{wd\_capped}\SpecialCharTok{$}\NormalTok{quality }\OtherTok{\textless{}{-}} \FunctionTok{squish}\NormalTok{(wd\_capped}\SpecialCharTok{$}\NormalTok{quality, }\FunctionTok{quantile}\NormalTok{(wd\_capped}\SpecialCharTok{$}\NormalTok{quality, }\FunctionTok{c}\NormalTok{(.}\DecValTok{05}\NormalTok{, .}\DecValTok{95}\NormalTok{))) }

\CommentTok{\# We compare the dimensions of the dataset before and after outlier removal}

\FunctionTok{dim}\NormalTok{(wine\_data)}
\end{Highlighting}
\end{Shaded}

\begin{verbatim}
## [1] 6497   12
\end{verbatim}

\begin{Shaded}
\begin{Highlighting}[]
\FunctionTok{dim}\NormalTok{(wd\_capped)}
\end{Highlighting}
\end{Shaded}

\begin{verbatim}
## [1] 6497   12
\end{verbatim}

\begin{Shaded}
\begin{Highlighting}[]
\FunctionTok{dim}\NormalTok{(wine\_data)[}\DecValTok{1}\NormalTok{] }\SpecialCharTok{{-}} \FunctionTok{dim}\NormalTok{(wd\_capped)[}\DecValTok{1}\NormalTok{]}
\end{Highlighting}
\end{Shaded}

\begin{verbatim}
## [1] 0
\end{verbatim}

\begin{Shaded}
\begin{Highlighting}[]
\FunctionTok{boxplot}\NormalTok{(wine\_data)}
\end{Highlighting}
\end{Shaded}

\includegraphics{Notebook_1_files/figure-latex/unnamed-chunk-2-1.pdf}

\begin{Shaded}
\begin{Highlighting}[]
\FunctionTok{boxplot}\NormalTok{(wd\_capped)}
\end{Highlighting}
\end{Shaded}

\includegraphics{Notebook_1_files/figure-latex/unnamed-chunk-2-2.pdf}

\begin{Shaded}
\begin{Highlighting}[]
\NormalTok{wine\_data }\OtherTok{\textless{}{-}}\NormalTok{ wd\_capped}
\end{Highlighting}
\end{Shaded}

\begin{Shaded}
\begin{Highlighting}[]
\CommentTok{\# pairs(wine\_data)}
\end{Highlighting}
\end{Shaded}

\hypertarget{summary-of-the-data}{%
\subsection{Summary of the data}\label{summary-of-the-data}}

\begin{Shaded}
\begin{Highlighting}[]
\FunctionTok{summary}\NormalTok{(wine\_data)}
\end{Highlighting}
\end{Shaded}

\begin{verbatim}
##  fixed.acidity    volatile.acidity  citric.acid    residual.sugar  
##  Min.   : 3.800   Min.   :0.1600   Min.   :0.050   Min.   : 1.200  
##  1st Qu.: 6.400   1st Qu.:0.2300   1st Qu.:0.250   1st Qu.: 1.800  
##  Median : 7.000   Median :0.2900   Median :0.310   Median : 3.000  
##  Mean   : 7.215   Mean   :0.3345   Mean   :0.315   Mean   : 5.334  
##  3rd Qu.: 7.700   3rd Qu.:0.4000   3rd Qu.:0.390   3rd Qu.: 8.100  
##  Max.   :15.900   Max.   :0.6700   Max.   :0.560   Max.   :15.000  
##    chlorides       free.sulfur.dioxide total.sulfur.dioxide    density      
##  Min.   :0.02800   Min.   : 6.00       Min.   : 19          Min.   :0.9899  
##  1st Qu.:0.03800   1st Qu.:17.00       1st Qu.: 77          1st Qu.:0.9923  
##  Median :0.04700   Median :29.00       Median :118          Median :0.9949  
##  Mean   :0.05327   Mean   :30.03       Mean   :115          Mean   :0.9947  
##  3rd Qu.:0.06500   3rd Qu.:41.00       3rd Qu.:156          3rd Qu.:0.9970  
##  Max.   :0.10200   Max.   :61.00       Max.   :206          Max.   :0.9994  
##        pH          sulphates         alcohol         quality    
##  Min.   :2.970   Min.   :0.3500   Min.   : 9.00   Min.   :5.00  
##  1st Qu.:3.110   1st Qu.:0.4300   1st Qu.: 9.50   1st Qu.:5.00  
##  Median :3.210   Median :0.5100   Median :10.30   Median :6.00  
##  Mean   :3.217   Mean   :0.5257   Mean   :10.48   Mean   :5.83  
##  3rd Qu.:3.320   3rd Qu.:0.6000   3rd Qu.:11.30   3rd Qu.:6.00  
##  Max.   :3.500   Max.   :0.7900   Max.   :12.70   Max.   :7.00
\end{verbatim}

\hypertarget{removal-of-outliers-interquartile-range}{%
\section{Removal of outliers (Interquartile
range)}\label{removal-of-outliers-interquartile-range}}

References:
\url{https://www.r-bloggers.com/2021/09/how-to-remove-outliers-in-r-3/}

To begin, we must first identify the outliers in a dataset; typically,
two methods are available. 1. z-scores 2. interquartile range

\hypertarget{description-of-methods}{%
\subsection{Description of methods}\label{description-of-methods}}

\hypertarget{z-score-method}{%
\subsection{z-score method}\label{z-score-method}}

The z-score indicates the number of standard deviations a given value
deviates from the mean. A z-score is calculated using the following
formula:

\[z = (X – \mu) / \sigma\] where:

\(X\) is a single raw data value

\(\mu\) is the population mean

\(\sigma\) is the population standard deviation

If an observation's z-score is less than -3 or larger than 3, it's
considered an outlier.

We implement this method below:

\begin{Shaded}
\begin{Highlighting}[]
\NormalTok{z\_scores }\OtherTok{\textless{}{-}} \FunctionTok{as.data.frame}\NormalTok{(}\FunctionTok{sapply}\NormalTok{(wine\_data, }\ControlFlowTok{function}\NormalTok{(wine\_data) (}\FunctionTok{abs}\NormalTok{(wine\_data}\SpecialCharTok{{-}}\FunctionTok{mean}\NormalTok{(wine\_data))}\SpecialCharTok{/}\FunctionTok{sd}\NormalTok{(wine\_data))))    }
\NormalTok{no\_outliers }\OtherTok{\textless{}{-}}\NormalTok{ z\_scores[}\SpecialCharTok{!}\FunctionTok{rowSums}\NormalTok{(z\_scores}\SpecialCharTok{\textgreater{}}\DecValTok{3}\NormalTok{), ]}
\CommentTok{\# head(no\_outliers)}
\FunctionTok{dim}\NormalTok{(wine\_data)}
\end{Highlighting}
\end{Shaded}

\begin{verbatim}
## [1] 6497   12
\end{verbatim}

\begin{Shaded}
\begin{Highlighting}[]
\FunctionTok{dim}\NormalTok{(no\_outliers)}
\end{Highlighting}
\end{Shaded}

\begin{verbatim}
## [1] 6369   12
\end{verbatim}

\begin{Shaded}
\begin{Highlighting}[]
\FunctionTok{dim}\NormalTok{(wine\_data)[}\DecValTok{1}\NormalTok{] }\SpecialCharTok{{-}} \FunctionTok{dim}\NormalTok{(no\_outliers)[}\DecValTok{1}\NormalTok{]}
\end{Highlighting}
\end{Shaded}

\begin{verbatim}
## [1] 128
\end{verbatim}

\begin{Shaded}
\begin{Highlighting}[]
\FunctionTok{boxplot}\NormalTok{(no\_outliers)}
\end{Highlighting}
\end{Shaded}

\includegraphics{Notebook_1_files/figure-latex/unnamed-chunk-5-1.pdf}

With the z-score method we see that we have 508 less observations in the
dataset

\hypertarget{interquartile-range-method}{%
\subsection{Interquartile range
method}\label{interquartile-range-method}}

In a dataset, it is the difference between the 75th percentile (Q3) and
the 25th percentile (Q1).

The interquartile range (IQR) is a measurement of the spread of values
in the middle 50\%.

If an observation is 1.5 times the interquartile range more than the
third quartile (Q3) or 1.5 times the interquartile range less than the
first quartile (Q1), it is considered an outlier (Q1).

\begin{Shaded}
\begin{Highlighting}[]
\NormalTok{varlist }\OtherTok{\textless{}{-}} \FunctionTok{names}\NormalTok{(wine\_data)}
\CommentTok{\# varlist \textless{}{-} varlist[{-}length(varlist)]}

\ControlFlowTok{for}\NormalTok{ (i }\ControlFlowTok{in}\NormalTok{ varlist) \{}
\NormalTok{    var }\OtherTok{\textless{}{-}} \FunctionTok{eval}\NormalTok{(}\FunctionTok{parse}\NormalTok{(}\AttributeTok{text =} \FunctionTok{paste0}\NormalTok{(}\StringTok{"wine\_data$"}\NormalTok{, i)))}
\NormalTok{    Q1 }\OtherTok{\textless{}{-}} \FunctionTok{quantile}\NormalTok{(var, }\FloatTok{0.25}\NormalTok{)}
\NormalTok{    Q3 }\OtherTok{\textless{}{-}} \FunctionTok{quantile}\NormalTok{(var, }\FloatTok{0.75}\NormalTok{)}
\NormalTok{    iqr }\OtherTok{\textless{}{-}} \FunctionTok{IQR}\NormalTok{(var)}
\NormalTok{    no\_outliers }\OtherTok{\textless{}{-}} \FunctionTok{subset}\NormalTok{(wine\_data, var }\SpecialCharTok{\textgreater{}}\NormalTok{ (Q1 }\SpecialCharTok{{-}} \FloatTok{1.5}\SpecialCharTok{*}\NormalTok{iqr) }\SpecialCharTok{\&}\NormalTok{ var }\SpecialCharTok{\textless{}}\NormalTok{ (Q3 }\SpecialCharTok{+} \FloatTok{1.5}\SpecialCharTok{*}\NormalTok{iqr) )}
    
\NormalTok{\}}

\CommentTok{\# We compare the dimensions of the dataset before and after outlier removal}
\FunctionTok{dim}\NormalTok{(wine\_data)}
\end{Highlighting}
\end{Shaded}

\begin{verbatim}
## [1] 6497   12
\end{verbatim}

\begin{Shaded}
\begin{Highlighting}[]
\FunctionTok{dim}\NormalTok{(no\_outliers)}
\end{Highlighting}
\end{Shaded}

\begin{verbatim}
## [1] 6497   12
\end{verbatim}

\begin{Shaded}
\begin{Highlighting}[]
\FunctionTok{dim}\NormalTok{(wine\_data)[}\DecValTok{1}\NormalTok{] }\SpecialCharTok{{-}} \FunctionTok{dim}\NormalTok{(no\_outliers)[}\DecValTok{1}\NormalTok{]}
\end{Highlighting}
\end{Shaded}

\begin{verbatim}
## [1] 0
\end{verbatim}

\begin{Shaded}
\begin{Highlighting}[]
\FunctionTok{boxplot}\NormalTok{(no\_outliers)}
\end{Highlighting}
\end{Shaded}

\includegraphics{Notebook_1_files/figure-latex/unnamed-chunk-6-1.pdf}
With the z-score method we see that we have 228 less observations in the
dataset

We should probably figure out which method we want to use at some point.

\hypertarget{chapter-6---linear-model-selection-and-regularization}{%
\section{Chapter 6 - Linear model selection and
regularization}\label{chapter-6---linear-model-selection-and-regularization}}

\hypertarget{subset-selection-methods}{%
\subsection{Subset selection methods}\label{subset-selection-methods}}

\hypertarget{best-subset-selection}{%
\subsubsection{Best Subset Selection}\label{best-subset-selection}}

\begin{Shaded}
\begin{Highlighting}[]
\FunctionTok{library}\NormalTok{(leaps)}
\end{Highlighting}
\end{Shaded}

\begin{verbatim}
## Warning: package 'leaps' was built under R version 4.2.2
\end{verbatim}

\begin{Shaded}
\begin{Highlighting}[]
\CommentTok{\# Perform best subset selection using the regsubsets() function included}
\CommentTok{\# in the leaps library}
\NormalTok{regfit.full }\OtherTok{\textless{}{-}} \FunctionTok{regsubsets}\NormalTok{(quality }\SpecialCharTok{\textasciitilde{}}\NormalTok{ ., wine\_data, }\AttributeTok{nvmax =} \DecValTok{11}\NormalTok{)}

\CommentTok{\# Summary of the full best subset selection model, the models with 1{-}11 variables}
\CommentTok{\# are shown below}
\CommentTok{\# shows the best set of variables for each model size}
\FunctionTok{summary}\NormalTok{(regfit.full)}
\end{Highlighting}
\end{Shaded}

\begin{verbatim}
## Subset selection object
## Call: regsubsets.formula(quality ~ ., wine_data, nvmax = 11)
## 11 Variables  (and intercept)
##                      Forced in Forced out
## fixed.acidity            FALSE      FALSE
## volatile.acidity         FALSE      FALSE
## citric.acid              FALSE      FALSE
## residual.sugar           FALSE      FALSE
## chlorides                FALSE      FALSE
## free.sulfur.dioxide      FALSE      FALSE
## total.sulfur.dioxide     FALSE      FALSE
## density                  FALSE      FALSE
## pH                       FALSE      FALSE
## sulphates                FALSE      FALSE
## alcohol                  FALSE      FALSE
## 1 subsets of each size up to 11
## Selection Algorithm: exhaustive
##           fixed.acidity volatile.acidity citric.acid residual.sugar chlorides
## 1  ( 1 )  " "           " "              " "         " "            " "      
## 2  ( 1 )  " "           "*"              " "         " "            " "      
## 3  ( 1 )  " "           "*"              " "         " "            " "      
## 4  ( 1 )  " "           "*"              " "         "*"            " "      
## 5  ( 1 )  " "           "*"              " "         "*"            " "      
## 6  ( 1 )  " "           "*"              " "         "*"            " "      
## 7  ( 1 )  " "           "*"              " "         "*"            " "      
## 8  ( 1 )  " "           "*"              " "         "*"            " "      
## 9  ( 1 )  "*"           "*"              " "         "*"            " "      
## 10  ( 1 ) "*"           "*"              "*"         "*"            " "      
## 11  ( 1 ) "*"           "*"              "*"         "*"            "*"      
##           free.sulfur.dioxide total.sulfur.dioxide density pH  sulphates
## 1  ( 1 )  " "                 " "                  " "     " " " "      
## 2  ( 1 )  " "                 " "                  " "     " " " "      
## 3  ( 1 )  " "                 " "                  " "     " " "*"      
## 4  ( 1 )  " "                 " "                  " "     " " "*"      
## 5  ( 1 )  " "                 "*"                  " "     " " "*"      
## 6  ( 1 )  "*"                 "*"                  " "     " " "*"      
## 7  ( 1 )  "*"                 "*"                  " "     "*" "*"      
## 8  ( 1 )  "*"                 "*"                  "*"     "*" "*"      
## 9  ( 1 )  "*"                 "*"                  "*"     "*" "*"      
## 10  ( 1 ) "*"                 "*"                  "*"     "*" "*"      
## 11  ( 1 ) "*"                 "*"                  "*"     "*" "*"      
##           alcohol
## 1  ( 1 )  "*"    
## 2  ( 1 )  "*"    
## 3  ( 1 )  "*"    
## 4  ( 1 )  "*"    
## 5  ( 1 )  "*"    
## 6  ( 1 )  "*"    
## 7  ( 1 )  "*"    
## 8  ( 1 )  "*"    
## 9  ( 1 )  "*"    
## 10  ( 1 ) "*"    
## 11  ( 1 ) "*"
\end{verbatim}

We now check to the different fitting criterion to see which model is
best

\begin{Shaded}
\begin{Highlighting}[]
\NormalTok{reg.summary }\OtherTok{=} \FunctionTok{summary}\NormalTok{(regfit.full)}
\end{Highlighting}
\end{Shaded}

\begin{Shaded}
\begin{Highlighting}[]
\FunctionTok{par}\NormalTok{(}\AttributeTok{mfrow =} \FunctionTok{c}\NormalTok{(}\DecValTok{1}\NormalTok{, }\DecValTok{3}\NormalTok{))}
\FunctionTok{plot}\NormalTok{(reg.summary}\SpecialCharTok{$}\NormalTok{adjr2, }\AttributeTok{xlab =} \StringTok{"Number of Variables"}\NormalTok{ ,}
\AttributeTok{ylab =} \StringTok{"Adjusted RSq"}\NormalTok{ , }\AttributeTok{type =} \StringTok{"l"}\NormalTok{)}

\FunctionTok{plot}\NormalTok{ (reg.summary}\SpecialCharTok{$}\NormalTok{cp, }\AttributeTok{xlab =} \StringTok{"Number of Variables"}\NormalTok{ ,}
\AttributeTok{ylab =} \StringTok{"Cp"}\NormalTok{ , }\AttributeTok{type =} \StringTok{"l"}\NormalTok{)}

\FunctionTok{plot}\NormalTok{ (reg.summary}\SpecialCharTok{$}\NormalTok{bic, }\AttributeTok{xlab =} \StringTok{"Number of Variables"}\NormalTok{ ,}
\AttributeTok{ylab =} \StringTok{"BIC"}\NormalTok{ , }\AttributeTok{type =} \StringTok{"l"}\NormalTok{)}
\end{Highlighting}
\end{Shaded}

\includegraphics{Notebook_1_files/figure-latex/unnamed-chunk-9-1.pdf}

\hypertarget{find-the-ideal-number-of-predictors-using-the-different-criterion}{%
\subsection{Find the ideal number of predictors using the different
criterion}\label{find-the-ideal-number-of-predictors-using-the-different-criterion}}

\begin{Shaded}
\begin{Highlighting}[]
\CommentTok{\# Find the number of predictors that corresponds to the maximum adjusted Rsq val}
\FunctionTok{which.max}\NormalTok{(reg.summary}\SpecialCharTok{$}\NormalTok{adjr2)}
\end{Highlighting}
\end{Shaded}

\begin{verbatim}
## [1] 10
\end{verbatim}

\begin{Shaded}
\begin{Highlighting}[]
\CommentTok{\# Find the number of predictors that corresponds to the minimum adjusted Cp val}
\FunctionTok{which.min}\NormalTok{(reg.summary}\SpecialCharTok{$}\NormalTok{cp)}
\end{Highlighting}
\end{Shaded}

\begin{verbatim}
## [1] 10
\end{verbatim}

\begin{Shaded}
\begin{Highlighting}[]
\CommentTok{\# Find the number of predictors that corresponds to the minimum adjusted BIC val}
\FunctionTok{which.min}\NormalTok{(reg.summary}\SpecialCharTok{$}\NormalTok{bic)}
\end{Highlighting}
\end{Shaded}

\begin{verbatim}
## [1] 9
\end{verbatim}

\hypertarget{plot-which-models-indicate-the-smallest-statistic}{%
\subsection{Plot which models indicate the smallest
statistic}\label{plot-which-models-indicate-the-smallest-statistic}}

\begin{Shaded}
\begin{Highlighting}[]
\FunctionTok{plot}\NormalTok{(regfit.full, }\AttributeTok{scale =} \StringTok{"adjr2"}\NormalTok{)}
\end{Highlighting}
\end{Shaded}

\includegraphics{Notebook_1_files/figure-latex/unnamed-chunk-11-1.pdf}

\begin{Shaded}
\begin{Highlighting}[]
\FunctionTok{plot}\NormalTok{(regfit.full, }\AttributeTok{scale =} \StringTok{"Cp"}\NormalTok{)}
\end{Highlighting}
\end{Shaded}

\includegraphics{Notebook_1_files/figure-latex/unnamed-chunk-11-2.pdf}

\begin{Shaded}
\begin{Highlighting}[]
\FunctionTok{plot}\NormalTok{(regfit.full, }\AttributeTok{scale =} \StringTok{"bic"}\NormalTok{)}
\end{Highlighting}
\end{Shaded}

\includegraphics{Notebook_1_files/figure-latex/unnamed-chunk-11-3.pdf}

\hypertarget{see-the-coefficients-associated-with-the-best-model}{%
\subsection{See the coefficients associated with the best
model}\label{see-the-coefficients-associated-with-the-best-model}}

\hypertarget{choosing-among-models-using-the-validation-set-approach-and-cross-validation}{%
\subsection{Choosing among models using the validation-set approach and
cross-validation}\label{choosing-among-models-using-the-validation-set-approach-and-cross-validation}}

\hypertarget{cross-validation}{%
\subsubsection{Cross validation}\label{cross-validation}}

\begin{Shaded}
\begin{Highlighting}[]
\FunctionTok{set.seed}\NormalTok{(}\DecValTok{1}\NormalTok{)}
\NormalTok{train }\OtherTok{\textless{}{-}} \FunctionTok{sample}\NormalTok{(}\FunctionTok{c}\NormalTok{(}\ConstantTok{TRUE}\NormalTok{, }\ConstantTok{FALSE}\NormalTok{), }\FunctionTok{nrow}\NormalTok{(wine\_data), }\AttributeTok{replace =} \ConstantTok{TRUE}\NormalTok{)}
\NormalTok{test }\OtherTok{\textless{}{-}}\NormalTok{ (}\SpecialCharTok{!}\NormalTok{train)}
\end{Highlighting}
\end{Shaded}

\begin{Shaded}
\begin{Highlighting}[]
\NormalTok{regfit.best }\OtherTok{\textless{}{-}} \FunctionTok{regsubsets}\NormalTok{(quality }\SpecialCharTok{\textasciitilde{}}\NormalTok{ ., }\AttributeTok{data =}\NormalTok{ wine\_data[train, ], }\AttributeTok{nvmax =} \DecValTok{11}\NormalTok{)}
\NormalTok{test.mat }\OtherTok{\textless{}{-}} \FunctionTok{model.matrix}\NormalTok{(quality }\SpecialCharTok{\textasciitilde{}}\NormalTok{ ., }\AttributeTok{data =}\NormalTok{ wine\_data[test, ])}
\end{Highlighting}
\end{Shaded}

\begin{Shaded}
\begin{Highlighting}[]
\NormalTok{val.errors }\OtherTok{\textless{}{-}} \FunctionTok{rep}\NormalTok{(}\DecValTok{0}\NormalTok{, }\DecValTok{11}\NormalTok{)}


\ControlFlowTok{for}\NormalTok{ (i }\ControlFlowTok{in} \DecValTok{1}\SpecialCharTok{:}\DecValTok{11}\NormalTok{) \{}
\NormalTok{    coefi }\OtherTok{\textless{}{-}} \FunctionTok{coef}\NormalTok{(regfit.best, }\AttributeTok{id =}\NormalTok{ i)}
\NormalTok{    pred }\OtherTok{\textless{}{-}}\NormalTok{ test.mat[, }\FunctionTok{names}\NormalTok{(coefi)] }\SpecialCharTok{\%*\%}\NormalTok{ coefi}
\NormalTok{    val.errors[i] }\OtherTok{\textless{}{-}} \FunctionTok{mean}\NormalTok{((wine\_data}\SpecialCharTok{$}\NormalTok{quality[test] }\SpecialCharTok{{-}}\NormalTok{ pred)}\SpecialCharTok{\^{}}\DecValTok{2}\NormalTok{)}
    
\NormalTok{\}}
\end{Highlighting}
\end{Shaded}

\begin{Shaded}
\begin{Highlighting}[]
\NormalTok{val.errors}
\end{Highlighting}
\end{Shaded}

\begin{verbatim}
##  [1] 0.4148673 0.3828393 0.3746997 0.3705519 0.3691098 0.3655426 0.3650684
##  [8] 0.3639048 0.3616509 0.3616630 0.3618294
\end{verbatim}

\begin{Shaded}
\begin{Highlighting}[]
\FunctionTok{which.min}\NormalTok{(val.errors)}
\end{Highlighting}
\end{Shaded}

\begin{verbatim}
## [1] 9
\end{verbatim}

We find that the best model is on that contains 10 variables. Therefore
we choose to use this model in our analysis.

\begin{Shaded}
\begin{Highlighting}[]
\NormalTok{predict.regsubsets }\OtherTok{\textless{}{-}} \ControlFlowTok{function}\NormalTok{(object, newdata, id , ...) \{}
\NormalTok{form }\OtherTok{\textless{}{-}} \FunctionTok{as.formula}\NormalTok{(object}\SpecialCharTok{$}\NormalTok{call[[}\DecValTok{2}\NormalTok{]])}
\NormalTok{mat }\OtherTok{\textless{}{-}} \FunctionTok{model.matrix}\NormalTok{(form, newdata )}
\NormalTok{coefi }\OtherTok{\textless{}{-}} \FunctionTok{coef}\NormalTok{(object, }\AttributeTok{id =}\NormalTok{ id)}
\NormalTok{xvars }\OtherTok{\textless{}{-}} \FunctionTok{names}\NormalTok{ (coefi)}
\NormalTok{mat[, xvars ] }\SpecialCharTok{\%*\%}\NormalTok{ coefi}
\NormalTok{\}}
\end{Highlighting}
\end{Shaded}

\begin{Shaded}
\begin{Highlighting}[]
\NormalTok{regfit.best }\OtherTok{\textless{}{-}} \FunctionTok{regsubsets}\NormalTok{(quality }\SpecialCharTok{\textasciitilde{}}\NormalTok{ ., }\AttributeTok{data =}\NormalTok{ wine\_data, }\AttributeTok{nvmax =} \DecValTok{11}\NormalTok{)}
\FunctionTok{coef}\NormalTok{(regfit.best, }\DecValTok{10}\NormalTok{)}
\end{Highlighting}
\end{Shaded}

\begin{verbatim}
##          (Intercept)        fixed.acidity     volatile.acidity 
##         68.960350471          0.074094737         -1.204312818 
##          citric.acid       residual.sugar  free.sulfur.dioxide 
##         -0.132337321          0.045462743          0.005868213 
## total.sulfur.dioxide              density                   pH 
##         -0.002382853        -68.228268139          0.504158979 
##            sulphates              alcohol 
##          0.883924623          0.230180918
\end{verbatim}

Now try to choose amongst the models of different sizes using
cross-validation (k-fold) 10-folds

Must perform best subset selection with each of the k training sets.

\begin{Shaded}
\begin{Highlighting}[]
\NormalTok{k }\OtherTok{\textless{}{-}} \DecValTok{5}
\NormalTok{n }\OtherTok{\textless{}{-}} \FunctionTok{nrow}\NormalTok{(wine\_data)}
\FunctionTok{set.seed}\NormalTok{(}\DecValTok{1}\NormalTok{)}
\NormalTok{folds }\OtherTok{\textless{}{-}} \FunctionTok{sample}\NormalTok{(}\FunctionTok{rep}\NormalTok{(}\DecValTok{1}\SpecialCharTok{:}\NormalTok{k, }\AttributeTok{length =}\NormalTok{ n))}
\NormalTok{cv.errors }\OtherTok{\textless{}{-}} \FunctionTok{matrix}\NormalTok{(}\ConstantTok{NA}\NormalTok{, k, }\DecValTok{11}\NormalTok{, }\AttributeTok{dimnames =} \FunctionTok{list}\NormalTok{(}\ConstantTok{NULL}\NormalTok{, }\FunctionTok{paste}\NormalTok{(}\DecValTok{1}\SpecialCharTok{:}\DecValTok{11}\NormalTok{)))}
\end{Highlighting}
\end{Shaded}

\begin{Shaded}
\begin{Highlighting}[]
\ControlFlowTok{for}\NormalTok{ (j }\ControlFlowTok{in} \DecValTok{1}\SpecialCharTok{:}\NormalTok{k) \{}
\NormalTok{    best.fit }\OtherTok{\textless{}{-}} \FunctionTok{regsubsets}\NormalTok{(quality }\SpecialCharTok{\textasciitilde{}}\NormalTok{ ., }\AttributeTok{data =}\NormalTok{ wine\_data[folds }\SpecialCharTok{!=}\NormalTok{ j, ], }\AttributeTok{nvmax =} \DecValTok{11}\NormalTok{)}
    
    \ControlFlowTok{for}\NormalTok{ (i }\ControlFlowTok{in} \DecValTok{1}\SpecialCharTok{:}\DecValTok{11}\NormalTok{) \{}
\NormalTok{        pred }\OtherTok{\textless{}{-}} \FunctionTok{predict}\NormalTok{(best.fit, wine\_data[folds }\SpecialCharTok{==}\NormalTok{ j, ], }\AttributeTok{id =}\NormalTok{ i)}
\NormalTok{        cv.errors[j, i] }\OtherTok{\textless{}{-}} \FunctionTok{mean}\NormalTok{((wine\_data}\SpecialCharTok{$}\NormalTok{quality[folds }\SpecialCharTok{==}\NormalTok{ j] }\SpecialCharTok{{-}}\NormalTok{ pred)}\SpecialCharTok{\^{}}\DecValTok{2}\NormalTok{)}
\NormalTok{    \}}
\NormalTok{\}}
\end{Highlighting}
\end{Shaded}

\begin{Shaded}
\begin{Highlighting}[]
\NormalTok{mean.cv.errors }\OtherTok{\textless{}{-}} \FunctionTok{apply}\NormalTok{(cv.errors, }\DecValTok{2}\NormalTok{, mean)}
\NormalTok{mean.cv.errors}
\end{Highlighting}
\end{Shaded}

\begin{verbatim}
##         1         2         3         4         5         6         7         8 
## 0.4136115 0.3841631 0.3771888 0.3727125 0.3712452 0.3674260 0.3677570 0.3667030 
##         9        10        11 
## 0.3648552 0.3649498 0.3649542
\end{verbatim}

\begin{Shaded}
\begin{Highlighting}[]
\FunctionTok{min}\NormalTok{(mean.cv.errors)}
\end{Highlighting}
\end{Shaded}

\begin{verbatim}
## [1] 0.3648552
\end{verbatim}

We see that cross validation selects a 9 variable model.

We now perform best subset selection on the full data set in order to
obtain this model

\begin{Shaded}
\begin{Highlighting}[]
\NormalTok{reg.best }\OtherTok{\textless{}{-}} \FunctionTok{regsubsets}\NormalTok{(quality }\SpecialCharTok{\textasciitilde{}}\NormalTok{ ., }\AttributeTok{data =}\NormalTok{ wine\_data, }\AttributeTok{nvmax =} \DecValTok{11}\NormalTok{)}
\FunctionTok{coef}\NormalTok{(reg.best, }\DecValTok{9}\NormalTok{)}
\end{Highlighting}
\end{Shaded}

\begin{verbatim}
##          (Intercept)        fixed.acidity     volatile.acidity 
##         70.003827333          0.069246882         -1.153656143 
##       residual.sugar  free.sulfur.dioxide total.sulfur.dioxide 
##          0.045698051          0.005878583         -0.002447426 
##              density                   pH            sulphates 
##        -69.311834099          0.520147565          0.874475435 
##              alcohol 
##          0.227325902
\end{verbatim}

\hypertarget{ridge-regression-and-the-lasso}{%
\subsection{Ridge-regression and the
lasso}\label{ridge-regression-and-the-lasso}}

\hypertarget{ridge-regression}{%
\subsection{Ridge-regression}\label{ridge-regression}}

\begin{Shaded}
\begin{Highlighting}[]
\FunctionTok{set.seed}\NormalTok{(}\DecValTok{1}\NormalTok{)}
\NormalTok{x }\OtherTok{\textless{}{-}} \FunctionTok{model.matrix}\NormalTok{(quality }\SpecialCharTok{\textasciitilde{}}\NormalTok{ ., wine\_data)[, }\SpecialCharTok{{-}}\DecValTok{1}\NormalTok{]}
\NormalTok{y }\OtherTok{\textless{}{-}}\NormalTok{ wine\_data}\SpecialCharTok{$}\NormalTok{quality}
\NormalTok{train }\OtherTok{\textless{}{-}} \FunctionTok{sample}\NormalTok{(}\DecValTok{1}\SpecialCharTok{:}\FunctionTok{nrow}\NormalTok{(x), }\FunctionTok{nrow}\NormalTok{(x)}\SpecialCharTok{/}\DecValTok{2}\NormalTok{)}
\NormalTok{test }\OtherTok{\textless{}{-}}\NormalTok{ (}\SpecialCharTok{{-}}\NormalTok{train)}
\NormalTok{y.test }\OtherTok{\textless{}{-}}\NormalTok{ y[test]}
\end{Highlighting}
\end{Shaded}

\begin{Shaded}
\begin{Highlighting}[]
\FunctionTok{library}\NormalTok{(glmnet)}
\end{Highlighting}
\end{Shaded}

\begin{verbatim}
## Warning: package 'glmnet' was built under R version 4.2.2
\end{verbatim}

\begin{verbatim}
## Loading required package: Matrix
\end{verbatim}

\begin{verbatim}
## Loaded glmnet 4.1-6
\end{verbatim}

\begin{Shaded}
\begin{Highlighting}[]
\NormalTok{grid }\OtherTok{\textless{}{-}} \DecValTok{10}\SpecialCharTok{\^{}}\FunctionTok{seq}\NormalTok{(}\DecValTok{10}\NormalTok{, }\SpecialCharTok{{-}}\DecValTok{2}\NormalTok{, }\AttributeTok{length =} \DecValTok{100}\NormalTok{)}
\end{Highlighting}
\end{Shaded}

\begin{Shaded}
\begin{Highlighting}[]
\NormalTok{ridge.mod }\OtherTok{\textless{}{-}} \FunctionTok{glmnet}\NormalTok{(x[train, ], y[train], }\AttributeTok{alpha =} \DecValTok{0}\NormalTok{, }\AttributeTok{lambda =}\NormalTok{ grid)}
\FunctionTok{plot}\NormalTok{(ridge.mod)}
\end{Highlighting}
\end{Shaded}

\includegraphics{Notebook_1_files/figure-latex/unnamed-chunk-25-1.pdf}

\begin{Shaded}
\begin{Highlighting}[]
\FunctionTok{set.seed}\NormalTok{(}\DecValTok{1}\NormalTok{)}
\NormalTok{cv.out }\OtherTok{\textless{}{-}} \FunctionTok{cv.glmnet}\NormalTok{(x[train, ], y[train], }\AttributeTok{alpha =} \DecValTok{0}\NormalTok{)}
\FunctionTok{plot}\NormalTok{(cv.out)}
\end{Highlighting}
\end{Shaded}

\includegraphics{Notebook_1_files/figure-latex/unnamed-chunk-26-1.pdf}

\begin{Shaded}
\begin{Highlighting}[]
\NormalTok{bestlam }\OtherTok{\textless{}{-}}\NormalTok{ cv.out}\SpecialCharTok{$}\NormalTok{lambda.min}
\NormalTok{lasso.pred }\OtherTok{\textless{}{-}} \FunctionTok{predict}\NormalTok{(ridge.mod, }\AttributeTok{s =}\NormalTok{ bestlam, }\AttributeTok{newx =}\NormalTok{ x[test, ])}
\FunctionTok{mean}\NormalTok{((lasso.pred }\SpecialCharTok{{-}}\NormalTok{ y.test)}\SpecialCharTok{\^{}}\DecValTok{2}\NormalTok{)}
\end{Highlighting}
\end{Shaded}

\begin{verbatim}
## [1] 0.3605988
\end{verbatim}

\begin{Shaded}
\begin{Highlighting}[]
\NormalTok{out }\OtherTok{\textless{}{-}} \FunctionTok{glmnet}\NormalTok{(x, y, }\AttributeTok{alpha =} \DecValTok{0}\NormalTok{, }\AttributeTok{lambda =}\NormalTok{ grid)}
\NormalTok{ridge.coef }\OtherTok{\textless{}{-}} \FunctionTok{predict}\NormalTok{(out, }\AttributeTok{type =} \StringTok{"coefficients"}\NormalTok{, }\AttributeTok{s =}\NormalTok{ bestlam)[}\DecValTok{1}\SpecialCharTok{:}\DecValTok{12}\NormalTok{, ]}
\NormalTok{ridge.coef}
\end{Highlighting}
\end{Shaded}

\begin{verbatim}
##          (Intercept)        fixed.acidity     volatile.acidity 
##         40.934526532          0.043956261         -1.134589970 
##          citric.acid       residual.sugar            chlorides 
##         -0.062270550          0.030537484         -1.173505845 
##  free.sulfur.dioxide total.sulfur.dioxide              density 
##          0.005148968         -0.002015011        -39.330410551 
##                   pH            sulphates              alcohol 
##          0.345771173          0.785362246          0.243205550
\end{verbatim}

\begin{Shaded}
\begin{Highlighting}[]
\NormalTok{ridge.coef[ridge.coef }\SpecialCharTok{!=} \DecValTok{0}\NormalTok{]}
\end{Highlighting}
\end{Shaded}

\begin{verbatim}
##          (Intercept)        fixed.acidity     volatile.acidity 
##         40.934526532          0.043956261         -1.134589970 
##          citric.acid       residual.sugar            chlorides 
##         -0.062270550          0.030537484         -1.173505845 
##  free.sulfur.dioxide total.sulfur.dioxide              density 
##          0.005148968         -0.002015011        -39.330410551 
##                   pH            sulphates              alcohol 
##          0.345771173          0.785362246          0.243205550
\end{verbatim}

\hypertarget{the-lasso}{%
\subsection{The lasso}\label{the-lasso}}

\begin{Shaded}
\begin{Highlighting}[]
\FunctionTok{set.seed}\NormalTok{(}\DecValTok{1}\NormalTok{)}
\NormalTok{x }\OtherTok{\textless{}{-}} \FunctionTok{model.matrix}\NormalTok{(quality }\SpecialCharTok{\textasciitilde{}}\NormalTok{ ., wine\_data)[, }\SpecialCharTok{{-}}\DecValTok{1}\NormalTok{]}
\NormalTok{y }\OtherTok{\textless{}{-}}\NormalTok{ wine\_data}\SpecialCharTok{$}\NormalTok{quality}
\NormalTok{train }\OtherTok{\textless{}{-}} \FunctionTok{sample}\NormalTok{(}\DecValTok{1}\SpecialCharTok{:}\FunctionTok{nrow}\NormalTok{(x), }\FunctionTok{nrow}\NormalTok{(x)}\SpecialCharTok{/}\DecValTok{2}\NormalTok{)}
\NormalTok{test }\OtherTok{\textless{}{-}}\NormalTok{ (}\SpecialCharTok{{-}}\NormalTok{train)}
\NormalTok{y.test }\OtherTok{\textless{}{-}}\NormalTok{ y[test]}
\end{Highlighting}
\end{Shaded}

\begin{Shaded}
\begin{Highlighting}[]
\FunctionTok{library}\NormalTok{(glmnet)}
\NormalTok{grid }\OtherTok{\textless{}{-}} \DecValTok{10}\SpecialCharTok{\^{}}\FunctionTok{seq}\NormalTok{(}\DecValTok{10}\NormalTok{, }\SpecialCharTok{{-}}\DecValTok{2}\NormalTok{, }\AttributeTok{length =} \DecValTok{100}\NormalTok{)}
\end{Highlighting}
\end{Shaded}

\begin{Shaded}
\begin{Highlighting}[]
\NormalTok{lasso.mod }\OtherTok{\textless{}{-}} \FunctionTok{glmnet}\NormalTok{(x[train, ], y[train], }\AttributeTok{alpha =} \DecValTok{1}\NormalTok{, }\AttributeTok{lambda =}\NormalTok{ grid)}
\FunctionTok{plot}\NormalTok{(lasso.mod)}
\end{Highlighting}
\end{Shaded}

\begin{verbatim}
## Warning in regularize.values(x, y, ties, missing(ties), na.rm = na.rm):
## collapsing to unique 'x' values
\end{verbatim}

\includegraphics{Notebook_1_files/figure-latex/unnamed-chunk-32-1.pdf}

\begin{Shaded}
\begin{Highlighting}[]
\FunctionTok{set.seed}\NormalTok{(}\DecValTok{1}\NormalTok{)}
\NormalTok{cv.out }\OtherTok{\textless{}{-}} \FunctionTok{cv.glmnet}\NormalTok{(x[train, ], y[train], }\AttributeTok{alpha =} \DecValTok{1}\NormalTok{)}
\FunctionTok{plot}\NormalTok{(cv.out)}
\end{Highlighting}
\end{Shaded}

\includegraphics{Notebook_1_files/figure-latex/unnamed-chunk-33-1.pdf}

\begin{Shaded}
\begin{Highlighting}[]
\NormalTok{bestlam }\OtherTok{\textless{}{-}}\NormalTok{ cv.out}\SpecialCharTok{$}\NormalTok{lambda.min}
\NormalTok{lasso.pred }\OtherTok{\textless{}{-}} \FunctionTok{predict}\NormalTok{(lasso.mod, }\AttributeTok{s =}\NormalTok{ bestlam, }\AttributeTok{newx =}\NormalTok{ x[test, ])}
\FunctionTok{mean}\NormalTok{((lasso.pred }\SpecialCharTok{{-}}\NormalTok{ y.test)}\SpecialCharTok{\^{}}\DecValTok{2}\NormalTok{)}
\end{Highlighting}
\end{Shaded}

\begin{verbatim}
## [1] 0.3631217
\end{verbatim}

\begin{Shaded}
\begin{Highlighting}[]
\NormalTok{out }\OtherTok{\textless{}{-}} \FunctionTok{glmnet}\NormalTok{(x, y, }\AttributeTok{alpha =} \DecValTok{1}\NormalTok{, }\AttributeTok{lambda =}\NormalTok{ grid)}
\NormalTok{lasso.coef }\OtherTok{\textless{}{-}} \FunctionTok{predict}\NormalTok{(out, }\AttributeTok{type =} \StringTok{"coefficients"}\NormalTok{, }\AttributeTok{s =}\NormalTok{ bestlam)[}\DecValTok{1}\SpecialCharTok{:}\DecValTok{12}\NormalTok{, ]}
\NormalTok{lasso.coef}
\end{Highlighting}
\end{Shaded}

\begin{verbatim}
##          (Intercept)        fixed.acidity     volatile.acidity 
##          2.474306957          0.000000000         -1.189634457 
##          citric.acid       residual.sugar            chlorides 
##          0.000000000          0.014669290         -0.754664550 
##  free.sulfur.dioxide total.sulfur.dioxide              density 
##          0.003832357         -0.001368393          0.000000000 
##                   pH            sulphates              alcohol 
##          0.093997115          0.634427372          0.297747749
\end{verbatim}

\begin{Shaded}
\begin{Highlighting}[]
\NormalTok{lasso.coef[lasso.coef }\SpecialCharTok{!=} \DecValTok{0}\NormalTok{]}
\end{Highlighting}
\end{Shaded}

\begin{verbatim}
##          (Intercept)     volatile.acidity       residual.sugar 
##          2.474306957         -1.189634457          0.014669290 
##            chlorides  free.sulfur.dioxide total.sulfur.dioxide 
##         -0.754664550          0.003832357         -0.001368393 
##                   pH            sulphates              alcohol 
##          0.093997115          0.634427372          0.297747749
\end{verbatim}

\begin{Shaded}
\begin{Highlighting}[]
\NormalTok{lasso.coef[lasso.coef }\SpecialCharTok{==} \DecValTok{0}\NormalTok{]}
\end{Highlighting}
\end{Shaded}

\begin{verbatim}
## fixed.acidity   citric.acid       density 
##             0             0             0
\end{verbatim}

\hypertarget{princpal-components-regression}{%
\subsection{Princpal components
regression}\label{princpal-components-regression}}

\begin{Shaded}
\begin{Highlighting}[]
\FunctionTok{set.seed}\NormalTok{(}\DecValTok{1}\NormalTok{)}
\NormalTok{x }\OtherTok{\textless{}{-}} \FunctionTok{model.matrix}\NormalTok{(quality }\SpecialCharTok{\textasciitilde{}}\NormalTok{ ., wine\_data)[, }\SpecialCharTok{{-}}\DecValTok{1}\NormalTok{]}
\NormalTok{y }\OtherTok{\textless{}{-}}\NormalTok{ wine\_data}\SpecialCharTok{$}\NormalTok{quality}
\NormalTok{train }\OtherTok{\textless{}{-}} \FunctionTok{sample}\NormalTok{(}\DecValTok{1}\SpecialCharTok{:}\FunctionTok{nrow}\NormalTok{(x), }\FunctionTok{nrow}\NormalTok{(x)}\SpecialCharTok{/}\DecValTok{2}\NormalTok{)}
\NormalTok{test }\OtherTok{\textless{}{-}}\NormalTok{ (}\SpecialCharTok{{-}}\NormalTok{train)}
\NormalTok{y.test }\OtherTok{\textless{}{-}}\NormalTok{ y[test]}
\end{Highlighting}
\end{Shaded}

\begin{Shaded}
\begin{Highlighting}[]
\FunctionTok{library}\NormalTok{(pls)}
\end{Highlighting}
\end{Shaded}

\begin{verbatim}
## Warning: package 'pls' was built under R version 4.2.2
\end{verbatim}

\begin{verbatim}
## 
## Attaching package: 'pls'
\end{verbatim}

\begin{verbatim}
## The following object is masked from 'package:stats':
## 
##     loadings
\end{verbatim}

\begin{Shaded}
\begin{Highlighting}[]
\FunctionTok{set.seed}\NormalTok{(}\DecValTok{2}\NormalTok{)}
\NormalTok{pcr.fit }\OtherTok{\textless{}{-}} \FunctionTok{pcr}\NormalTok{(quality }\SpecialCharTok{\textasciitilde{}}\NormalTok{ ., }\AttributeTok{data =}\NormalTok{ wine\_data, }\AttributeTok{scale =} \ConstantTok{TRUE}\NormalTok{, }\AttributeTok{validation =} \StringTok{"CV"}\NormalTok{)}
\end{Highlighting}
\end{Shaded}

\begin{Shaded}
\begin{Highlighting}[]
\FunctionTok{summary}\NormalTok{(pcr.fit)}
\end{Highlighting}
\end{Shaded}

\begin{verbatim}
## Data:    X dimension: 6497 11 
##  Y dimension: 6497 1
## Fit method: svdpc
## Number of components considered: 11
## 
## VALIDATION: RMSEP
## Cross-validated using 10 random segments.
##        (Intercept)  1 comps  2 comps  3 comps  4 comps  5 comps  6 comps
## CV          0.7312   0.7252   0.6807   0.6651   0.6478   0.6327   0.6246
## adjCV       0.7312   0.7252   0.6807   0.6651   0.6477   0.6327   0.6245
##        7 comps  8 comps  9 comps  10 comps  11 comps
## CV      0.6243   0.6238   0.6045    0.6047    0.6038
## adjCV   0.6242   0.6238   0.6045    0.6046    0.6038
## 
## TRAINING: % variance explained
##          1 comps  2 comps  3 comps  4 comps  5 comps  6 comps  7 comps  8 comps
## X         30.154    53.23    67.55    76.24    81.72    87.12    91.63    94.88
## quality    1.628    13.39    17.34    21.61    25.23    27.18    27.28    27.41
##          9 comps  10 comps  11 comps
## X          97.81     99.71    100.00
## quality    31.86     31.86     32.07
\end{verbatim}

\begin{Shaded}
\begin{Highlighting}[]
\FunctionTok{validationplot}\NormalTok{(pcr.fit, }\AttributeTok{val.type =} \StringTok{"MSEP"}\NormalTok{)}
\end{Highlighting}
\end{Shaded}

\includegraphics{Notebook_1_files/figure-latex/unnamed-chunk-41-1.pdf}

\begin{Shaded}
\begin{Highlighting}[]
\FunctionTok{set.seed}\NormalTok{(}\DecValTok{1}\NormalTok{)}
\NormalTok{pcr.fit }\OtherTok{\textless{}{-}} \FunctionTok{pcr}\NormalTok{(quality }\SpecialCharTok{\textasciitilde{}}\NormalTok{ ., }\AttributeTok{data =}\NormalTok{ wine\_data, }\AttributeTok{scale =} \ConstantTok{TRUE}\NormalTok{, }\AttributeTok{validation =} \StringTok{"CV"}\NormalTok{)}
\FunctionTok{validationplot}\NormalTok{(pcr.fit, }\AttributeTok{val.type =} \StringTok{"MSEP"}\NormalTok{)}
\end{Highlighting}
\end{Shaded}

\includegraphics{Notebook_1_files/figure-latex/unnamed-chunk-42-1.pdf}

\begin{Shaded}
\begin{Highlighting}[]
\NormalTok{pcr.pred }\OtherTok{\textless{}{-}} \FunctionTok{predict}\NormalTok{(pcr.fit, x[test, ], }\AttributeTok{ncomp =} \DecValTok{5}\NormalTok{)}
\FunctionTok{mean}\NormalTok{((pcr.pred }\SpecialCharTok{{-}}\NormalTok{ y.test)}\SpecialCharTok{\^{}}\DecValTok{2}\NormalTok{)}
\end{Highlighting}
\end{Shaded}

\begin{verbatim}
## [1] 0.3942515
\end{verbatim}

\begin{Shaded}
\begin{Highlighting}[]
\NormalTok{pcr.fit }\OtherTok{\textless{}{-}} \FunctionTok{pcr}\NormalTok{(y }\SpecialCharTok{\textasciitilde{}}\NormalTok{ x, }\AttributeTok{scale =} \ConstantTok{TRUE}\NormalTok{, }\AttributeTok{ncomp =} \DecValTok{5}\NormalTok{)}
\FunctionTok{summary}\NormalTok{(pcr.fit)}
\end{Highlighting}
\end{Shaded}

\begin{verbatim}
## Data:    X dimension: 6497 11 
##  Y dimension: 6497 1
## Fit method: svdpc
## Number of components considered: 5
## TRAINING: % variance explained
##    1 comps  2 comps  3 comps  4 comps  5 comps
## X   30.154    53.23    67.55    76.24    81.72
## y    1.628    13.39    17.34    21.61    25.23
\end{verbatim}

\hypertarget{partial-least-squares}{%
\subsection{Partial least squares}\label{partial-least-squares}}

\begin{Shaded}
\begin{Highlighting}[]
\FunctionTok{set.seed}\NormalTok{(}\DecValTok{1}\NormalTok{)}
\NormalTok{x }\OtherTok{\textless{}{-}} \FunctionTok{model.matrix}\NormalTok{(quality }\SpecialCharTok{\textasciitilde{}}\NormalTok{ ., wine\_data)[, }\SpecialCharTok{{-}}\DecValTok{1}\NormalTok{]}
\NormalTok{y }\OtherTok{\textless{}{-}}\NormalTok{ wine\_data}\SpecialCharTok{$}\NormalTok{quality}
\NormalTok{train }\OtherTok{\textless{}{-}} \FunctionTok{sample}\NormalTok{(}\DecValTok{1}\SpecialCharTok{:}\FunctionTok{nrow}\NormalTok{(x), }\FunctionTok{nrow}\NormalTok{(x)}\SpecialCharTok{/}\DecValTok{2}\NormalTok{)}
\NormalTok{test }\OtherTok{\textless{}{-}}\NormalTok{ (}\SpecialCharTok{{-}}\NormalTok{train)}
\NormalTok{y.test }\OtherTok{\textless{}{-}}\NormalTok{ y[test]}
\end{Highlighting}
\end{Shaded}

\begin{Shaded}
\begin{Highlighting}[]
\NormalTok{pls.fit }\OtherTok{\textless{}{-}} \FunctionTok{plsr}\NormalTok{(quality }\SpecialCharTok{\textasciitilde{}}\NormalTok{ ., }\AttributeTok{data =}\NormalTok{ wine\_data, }\AttributeTok{subset =}\NormalTok{ train, }\AttributeTok{scale =} \ConstantTok{TRUE}\NormalTok{, }\AttributeTok{validation =} \StringTok{"CV"}\NormalTok{)}
\FunctionTok{summary}\NormalTok{(pls.fit)}
\end{Highlighting}
\end{Shaded}

\begin{verbatim}
## Data:    X dimension: 3248 11 
##  Y dimension: 3248 1
## Fit method: kernelpls
## Number of components considered: 11
## 
## VALIDATION: RMSEP
## Cross-validated using 10 random segments.
##        (Intercept)  1 comps  2 comps  3 comps  4 comps  5 comps  6 comps
## CV          0.7323   0.6456   0.6211   0.6130   0.6096   0.6083   0.6080
## adjCV       0.7323   0.6456   0.6211   0.6129   0.6095   0.6082   0.6079
##        7 comps  8 comps  9 comps  10 comps  11 comps
## CV      0.6072   0.6072   0.6072    0.6072    0.6072
## adjCV   0.6071   0.6071   0.6071    0.6071    0.6071
## 
## TRAINING: % variance explained
##          1 comps  2 comps  3 comps  4 comps  5 comps  6 comps  7 comps  8 comps
## X          22.67    38.37    58.38    70.97    78.71    84.36    84.86    88.20
## quality    22.46    28.36    30.26    31.07    31.37    31.44    31.66    31.66
##          9 comps  10 comps  11 comps
## X          91.91     96.16    100.00
## quality    31.66     31.66     31.66
\end{verbatim}

\begin{Shaded}
\begin{Highlighting}[]
\FunctionTok{validationplot}\NormalTok{(pls.fit, }\AttributeTok{val.type =} \StringTok{"MSEP"}\NormalTok{)}
\end{Highlighting}
\end{Shaded}

\includegraphics{Notebook_1_files/figure-latex/unnamed-chunk-47-1.pdf}

\begin{Shaded}
\begin{Highlighting}[]
\NormalTok{pls.pred }\OtherTok{\textless{}{-}} \FunctionTok{predict}\NormalTok{(pls.fit, x[test, ], }\AttributeTok{ncomp =} \DecValTok{1}\NormalTok{)}
\FunctionTok{mean}\NormalTok{((pls.pred }\SpecialCharTok{{-}}\NormalTok{ y.test)}\SpecialCharTok{\^{}}\DecValTok{2}\NormalTok{)}
\end{Highlighting}
\end{Shaded}

\begin{verbatim}
## [1] 0.4068395
\end{verbatim}

\begin{Shaded}
\begin{Highlighting}[]
\NormalTok{pls.fit }\OtherTok{\textless{}{-}} \FunctionTok{plsr}\NormalTok{(quality }\SpecialCharTok{\textasciitilde{}}\NormalTok{ ., }\AttributeTok{data =}\NormalTok{ wine\_data, }\AttributeTok{scale =} \ConstantTok{TRUE}\NormalTok{, }\AttributeTok{ncomp =} \DecValTok{1}\NormalTok{)}
\FunctionTok{summary}\NormalTok{(pls.fit)}
\end{Highlighting}
\end{Shaded}

\begin{verbatim}
## Data:    X dimension: 6497 11 
##  Y dimension: 6497 1
## Fit method: kernelpls
## Number of components considered: 1
## TRAINING: % variance explained
##          1 comps
## X          22.53
## quality    23.15
\end{verbatim}

\end{document}
